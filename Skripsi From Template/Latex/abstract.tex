%!TEX root = skripsi.tex
%
% Halaman Abstract
%
% @author  Andreas Febrian
% @version 1.00
%

\chapter*{ABSTRACT}

\vspace*{0.2cm}

\noindent \begin{tabular}{l l p{11.0cm}}
	Name&: & \penulis \\
	Program&: & \programEng \\
	Title&: & \judulInggris \\
\end{tabular} \\ 

\vspace*{0.5cm}

\noindent 
Textual Entailment is a research in NLP that aims to identify whether there is an entailment relation between two texts. Textual Entailment research has been developed in a variety of languages but it is rare for the Indonesian language. This study aimed to develop a corpus of Indonesian Textual Entailment with Co-training method, a semi-supervised learning method that has been used in the development of English Textual Entailment corpus. Wikipedia revision history is used as the data resources. At the end of the study, the corpus contains 1857 data that is generated automatically with 76\% accuracy. The results of this study show that the combination of Co-training method and the Wikipedia revision history data could potentially produce a good corpus of Indonesian Textual Entailment.\\


\vspace*{0.2cm}

\noindent Keywords: \\ 
\noindent Textual Entailment, Co-training, Wikipedia \textit{revision history}, corpus, Indonesian language\\

\newpage
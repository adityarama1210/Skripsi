%!TEX root = skripsi.tex
%
% Halaman Abstract
%
% @author  Andreas Febrian
% @version 1.00
%

\chapter*{ABSTRACT}

\vspace*{0.2cm}

\noindent \begin{tabular}{l l p{11.0cm}}
	Name&: & \penulis \\
	Program&: & \programEng \\
	Title&: & \judulInggris \\
\end{tabular} \\ 

\vspace*{0.5cm}

\noindent 
Cross Lingual Word Sense Disambiguation (\textit{CLSWD}) is one among the methods in solving word sense disambiguation problem in NLP field. This approach utilize a concept that a word can translated to many words depend on where that word appear. Limitation of data (\textit{sense tagged corpus} in Indonesian language) become one problem that hold the development of research in Indonesian WSD. In this research, CLWSD approach will be used to transfer sense from english sense tagged corpus into Indonesian by using parallel corpora. Result of this research is a sense tagged corpus in Indonesian language that will be tested by our implemented WSD system. Based on the result of the experiment, we could see that Indonesian sense tagged corpora has been built by the proposed method. Beside the sense tagged corpora, the WSD system itself also having performance above the baseline on some given target words and features.
\vspace*{0.2cm}

\noindent Keywords: \\ 
\noindent Cross Lingual, Word Sense Disambiguation

\newpage
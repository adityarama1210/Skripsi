%!TEX root = skripsi.tex
%-----------------------------------------------------------------------------%
\chapter{\babTiga}
%-----------------------------------------------------------------------------%
Bab ini akan menjelaskan gambaran proses penelitian secara keseluruhan yang terdiri dari \textit{word alignment} korpus paralel, peningkatan kualitas dan evaluasi \textit{word alignment}, pemindahan \textit{sense} dari korpus bahasa Inggris, dan sistem WSD yang akan diimplementasikan.

%-----------------------------------------------------------------------------%
\section{Penyelarasan Kata pada Korpus Paralel}
%-----------------------------------------------------------------------------%
Penjajaran kata pada korpus berbahasa Inggris dan Indonesia menggunakan \textit{tools word alignment} bernama Giza++. \textit{Tool} ini merupakan salah satu \textit{word alignment tools} pada \textit{statistical machine translation} (SMT) yang dapat digunakan untuk memasangkan kata-kata pada dua buah korpus atau lebih. Terdapat beberapa \textit{word alignment tools} lain seperti Berkeley \textit{aligner}, anymalign, dan lain-lain. Penyelarasan kata ini digunakan untuk kebutuhan pemindahan \textit{sense} dari kata bahasa Inggris ke kata dalam bahasa Indonesia.

Proses penyelarasan yang dilakukan dengan Giza++ meliputi tahap-tahap berikut:
\begin{enumerate}
	\item Mempersiapkan kedua buah \textit{file} yaitu korpus bahasa asal (\textit{source}) dan korpus bahasa tujuan (\textit{target}). Kedua \textit{file} ini berpasangan dalam setiap barisnya. Baris pertama dalam \textit{file} pertama berpasangan dengan baris pertama pada \textit{file} kedua sampai akhir baris pada kedua \textit{file}.
	\item Menghasilkan \textit{file} perbendaharaan kata dari kedua bahasa dan \textit{list} indeks perbendaharaan kata pada tiap kalimat yang sudah diselaraskan
	\item Menghasilkan \textit{cooccurence file} dari kosa kata dan pasangan kalimat tersebut
	\item Proses \textit{alignment} yang menghasilkan beberapa macam \textit{output file} 
\end{enumerate}

Terdapat satu buah \textit{output file} Giza++ yang berisi pasangan-pasangan kalimat dengan kata-kata yang sudah diselaraskan dengan translasinya dalam bahasa tujuan. Hasil ini merupakan \textit{best viterbi alignment} menurut Giza++.

Pada skenario \textit{alignment} dengan bahasa Indonesia sebagai \textit{source} dan bahasa Inggris sebagai \textit{target}, satu kata dalam bahasa Indonesia akan dipasangkan dengan tepat satu kata dalam bahasa Inggris.

%-----------------------------------------------------------------------------%
\section{Evaluasi \textit{Word Alignment}} \label{sec:pembentukanTdanH}
%-----------------------------------------------------------------------------%
\textit{Word alignment} hasil dari \textit{tool} Giza++ dievaluasi dengan menggunakan \textit{anotator} hasil \textit{alignment} dari \textit{anotator} yang akan ditujukan sebagai \textit{gold standard}. Nilai-nilai yang akan dihitung meliputi \textit{precision} (P), \textit{recall} (R), dan F-\textit{score}. Metode evaluasi keseluruhan meliputi:

\begin{enumerate}
	\item Pemilihan \textit{random sampling} sebanyak seratus buah pasangan kalimat
	\item Masing-masing \textit{anotator} memasang-masangkan kata yang tepat pada masing-masing pasangan kalimat, dengan asumsi bahwa anotasi manusia sebagai \textit{gold standard}
	\item Hasil anotasi manusia dan keluaran dari \textit{tool} Giza dibandingkan untuk mendapatkan ketiga nilai P, R, dan F-Score.
\end{enumerate}


%-------%
\section{Peningkatan Kualitas Hasil \textit{Alignment}}
%-----------------------------------------------------------------------------%
Proses peningkatan kualitas hasil alignment diperlukan untuk meminimalisir kesalahan pemasangan kata-kata pada proses sebelumnya. Permasalahan  yang terjadi adalah adanya pasangan-pasangan kata yang tidak benar seperti pada halnya kata "lapangan" yang  dipasangkan dengan kata dalam bahasa inggris \textit{field}, \textit{ground}, \textit{involved}, \textit{job}, \textit{program}, dan beberapa kata lainnya. Peningkatan kualitas \textit{alignment} ini dilakukan dengan memanfaatkan hasil \textit{inverse} \textit{alignment} antara bahasa Indonesia ke Inggris. Pemanfaatkan hasil \textit{alignment} korpus bahasa Inggris ke Indonesia akan menghasilkan pasangan-pasangan kata dengan tingkat kesalahan \textit{alignment} lebih kecil. Metode yang akan dilakukan adalah dengan memeriksa setiap pasangan kata dari bahasa Indonesia yang mana merupakan kata dalam bahasa Inggris, apakah kata tersebut memiliki pasangan dalam bahasa Indonesia yang sama dengan keluaran dari \textit{inverse alignment} Giza.

Pada kasus kata \textbf{lingkungan} dari hasil keluaran Giza memiliki pasangan kata:

\begin{enumerate}
	\item environment
	\item environmental
	\item neighborhood
	\item within
	\item environmentally
\end{enumerate}

Untuk setiap pasangan kata dalam bahasa Inggris tersebut, akan dilakukan pengecekan apa saja pasangan kata bahasa Indonesianya. Bila terdapat kata \textbf{lingkungan} dalam pasangan kata bahasa Indonesianya maka kata tersebut dianggap pasangan yang benar.

Kata \textbf{environment} memiliki pasangan dalam bahasa Indonesia:

\begin{enumerate}
	\item lingkungan
	\item lingkup
\end{enumerate}

Keberadaan kata \textbf{lingkungan} dari pasangan kata \textit{environment} mengakibatkan kata \textit{environment} dianggap sebagai pasangan kata yang benar dari \textit{lingkungan}. Proses ini dilakukan untuk setiap kata dalam bahasa Inggris yang merupakan pasangan kata dalam bahasa Indonesia.
%-----------------------------------------------------------------------------%
\section{Pemindahan Makna Kata} \label{sec:Pemindahan Makna Kata}
%-----------------------------------------------------------------------------%
Pemindahan makna kata dilakukan dengan dua buah \textit{sub-process} yang terdiri dari:
1. Pemasangan antar kalimat yang bersesuaian dengan kata-kata yang berpasangan. Pada contoh kata "halaman" yang berpasangan dengan "courtyard", maka pasangan kalimat "Aku bermain di halaman" akan dipasangkan dengan kalimat "I play at the courtyard".

2. Setelah pasangan kalimat-kalimat yang bersesuaian dipasangkan, kalimat dalam bahasa Inggris tersebut diberikan \textit{sense} dengan menggunakan tool IMS.

3. \textit{Sense} dari kata yang menjadi \textit{target} tersebut kemudian dipindahkan ke kata yang bersesuaian pada bahasa Indonesianya di kalimat tersebut. Bila "courtyard" memiliki \textit{sense} yang artinya adalah "halaman rumah", maka "halaman" pada kalimat "Aku bermain di halaman" memiliki \textit{sense} "halaman rumah".

\section{Sistem WSD} \label{sec:Sistem WSD}
%-----------------------------------------------------------------------------%
Sistem WSD yang dibangun adalah dengan menggunakan pendekatan \textit{supervised learning}. Hasil dari pemindahan makna kata akan digunakan sebagai \textit{training} dan \textit{testing} data untuk menguji performa dari sistem yang dibangun. \textit{Classifier} yang digunakan dalam sistem \textit{WSD} ini adalah SVM. Pengujian dilakukan dengan menggunakan beberapa fitur seperti \textit{bag of words}, \textit{POS Tag}, dan \textit{word embedding}. Fitur \textit{bag of words} menggunakan \textit{window} sebanyak dua buah kata kanan dan kiri kata tujuan sebagai kata konteks. \textit{POS Tag} dan vektor \textit{word embedding} juga akan diimplementasikan pada penelitian ini. Performa dari sistem WSD akan dilihat berdasarkan perhitungan F1-score \textit{micro} dari hasil klasifikasi.
%!TEX root = skripsi.tex
%-----------------------------------------------------------------------------%
\chapter{\babEmpat} \label{implementasi}
%-----------------------------------------------------------------------------%
Bab ini akan menjelaskan perihal implementasi dari pemindahan \textit{sense} dan sistem \textit{WSD} yang dibuat.


\section{Peningkatan Kualitas \textit{Alignment}}

Hasil dari \textit{alignment} kata yang dilakukan Giza masih menghasilkan pasangan-pasangan kata yang tidak tepat. Untuk mengurangi jumlah pasangan kata yang salah tersebut, dilakukan \textit{enhancement} dengan dua kali proses \textit{alignment} baik itu dari Indonesia ke Inggris maupun sebaliknya.

Konsep dari proses yang dilakukan adalah melakukan validasi terhadap kata-kata yang berpasangan  dari kedua korpus. Pertama, setiap kata dalam bahasa Indonesia dikumpulkan terlebih dahulu dengan setiap pasangan kata bahasa Inggrisnya (satu kata bisa memiliki lebih dari satu pasangan). Proses selanjutnya adalah melakukan pengumpulan yang serupa terhadap kata dalam bahasa Inggris dengan pasangan kata dalam bahasa Indonesianya. Proses validasi dilakukan dengan cara:

\begin{enumerate}
	\item Untuk setiap kata di bahasa Indonesia semisal kata "kali"
	\item Lakukan pengecekan terhadap setiap pasangan kata di bahasa Inggris dari "kali" misalkan "time", "river", "fire"
	\item Jika pada kamus \textit{enhancement} "time" dipasangkan dengan "kali", dan "waktu" maka kata "time" merupakan pasangan yang dianggap benar. Pada kasus kata "fire", bila pasangan bahasa Indonesianya adalah "api" dan "tungku", maka kata "fire" dianggap bukan pasangan yang tepat dengan "kali" karena tidak terdapat pasangan "fire -> kali".
\end{enumerate} 

\begin{lstlisting}[language=Python, caption={Word Alignment Enhancement}, label={word-alignment-enhancement}]

dict_id = {}
dict_en = {}

# masukan setiap kosa kata bahasa Indonesia ke dalam dict_id sebagai key dan kumpulan pasangan kata bahasa inggrisnya sebagai value
# proses yang sama dilakukan untuk dict_en dengan kosa kata bahasa Inggris sebagai key dan kumpulan pasangan kata bahasa Indonesia sebagai value
# stop adalah list stopword yang didapat dari korpus nltk

# this section is for filtering which english word that has corresponding indo translation (bidirectional) from Giza output
for indo_word in dict_id.keys():
if indo_word not in dict_en:
# filtering so no same translation is entered, answer -> answer, jawaban -> jawaban
for en_word in dict_id[indo_word].keys():
if en_word in dict_en and indo_word in dict_en[en_word] and en_word not in stop:
if indo_word not in final_dictionary:
final_dictionary[indo_word] = { en_word: dict_en[word_en][word_id] }
else:
if en_word not in final_dictionary[indo_word]:
final_dictionary[indo_word][en_word] = dict_en[word_en][word_id]
\end{lstlisting}

%-----------------------------------------------------------------------------%
\section{Sistem WSD}
Sistem WSD dibangun dengan menggunakan \textit{supervised learning}. Pada sistem ini terdapat dua buah bagian utama yaitu pemilihan fitur serta \textit{classifier} dan evaluasi hasil \textit{classifier}.

\subsection{Pemilihan Fitur dan \textit{Classifier}}
Terdapat n buah fitur pada penelitian ini yang terdiri dari:

\begin{enumerate}
	\item Fitur \textit{bag of words} yang merupakan kata samping kanan dan kiri dengan \textit{window} sebanyak dua buah kata
	\item Fitur \textit{POS tagging}
	\item Vektor dari hasil \textit{word embedding}. Model \textit{word embedding} dilatih dengan menggunakan korpus dari Wikipedia bahasa Indonesia.
\end{enumerate}

\textit{Classifier} yang digunakan pada penelitian ini adalah SVM dengan \textit{library} Python yaitu Scikit dengan parameter \textit{default} berupa kernel linear dan C=1.

\subsection{Evaluasi Sistem}
Evaluasi dilakukan dengan \textit{cross validation} menggunakan perhitungan F1-score dari hasil klasifikasi yang dilakukan \textit{classifier}. \textit{Crossvalidation} dilakukan dengan interasi sebanyak tiga buah dengan perbandingan antara \textit{training} dan \textit{test set} sebesar 0,7:0,3.


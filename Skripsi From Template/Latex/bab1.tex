%!TEX root = skripsi.tex
%-----------------------------------------------------------------------------%
\chapter{\babSatu}
%-----------------------------------------------------------------------------%
Bab ini membahas mengenai latar belakang penelitian, perumusan masalah, tujuan dan manfaat penelitian, ruang lingkup penelitian, metodologi penelitian, serta sistematika penulisan.

%-----------------------------------------------------------------------------%
\section{Latar Belakang}
%-----------------------------------------------------------------------------%	

\textit{Task word sense disambiguation} merupakan pekerjaan yang mempunyai tujuan utama untuk mengenali \textit{sense}(makna) yang tepat dari sebuah kata. Sebagai manusia, pekerjaan ini termasuk relatif mudah karena kita dapat secara mengenal \textit{sense} dari sebuah kata berdasarkan makna dari konteks yang ada di sekelilingnya. Komputer pada dasarnya hanya dapat mengenali kata sebagai sebuah representasi data, sehingga perlu adanya \textit{modeling} agar komputer dapat mengenal dan membedakan makna antara satu kata dengan kata lainya. Sebuah contoh sederhana dari penentuan makna yang tepat dari sebuah kata dapat dilihat pada kalimat berikut.

(S1) : Setiap orang \textbf{bisa} melakukan tindak kriminal.

(S2) : Racun dari \textbf{bisa} ular itu sudah menyebar di tubuhnya.

(S3) : Ani sangat senang memakan \textbf{cokelat} pemberian Budi.

(S4) : Budi mempunyai sofa \textbf{cokelat} yang sangat mahal

Kata yang memiliki makna ganda (polisemi) pada kalimat satu dan kalimat dua adalah "bisa". kata "bisa" pada kedua kalimat tersebut memiliki makna yang berbeda, dimana kata "bisa" dalam kalimat satu menunjukan "kemampuan seseorang untuk melakukan sesuatu", dan kata "bisa" kalimat dua merujuk pada makna "racun yang berasal dari gigitan ular". Pada contoh permasalahan kalimat satu dan dua, penyelesaian relatif cukup mudah jika dilakukan dengan pendekatan \textit{POS Tagging}. Penggunaan bantuan \textit{task POS Tagging} tersebut dapat langsung membedakan bahwa "bisa" pada kalimat satu adalah kata kerja, sementara "bisa" kalimat kedua adalah objek. Kalimat ketiga dan keempat memberikan contoh lain dari makna yang berbeda pada kata yang sama berdasarkan konteks yang berbeda. Cokelat pada kalimat pertama bermakna makanan manis olahan dari hasil pohon cokelat sementara cokelat kalimat keempat merupakan keterangan warna dari objek sofa.

Pada umumnya, permasalahan ini dapat dikategorikan sebagai permasalahan klasifikasi. Diberikan sebuah \textit{resource}(seperti Wordnet) yang di dalamnya terdapat kata-kata beserta kumpulan maknanya masing-masing, makna kata mana yang paling tepat untuk suatu kata "x". Namun demikian, \textit{resource} yang dibutuhkan sebagai panduan makna kata tersebut harus memadai jika ingin digunakan untuk \textit{task} klasifikasi. Wordnet yang dimiliki untuk bahasa Indonesia masih tergolong kurang memadai karena jumlah \textit{synset} yang sedikit. Berdasarkan permasalahan tersebut, beberapa makna kata tambahan akan dipindahkan dari korpus yang memiliki \textit{resource} yang lebih mapan yaitu korpus bahasa Inggris dengan cara \textit{cross language sense transfering}.

Penggunaan dari \textit{word sense disambiguation} ini dapat bermanfaat untuk \textit{sub task} pada \textit{machine translation}, dimana satu buah kata dapat diterjemahkan menjadi beberapa kemungkinan kata yang berbeda berdasarkan makna yang dikandungnya. Mengacu pada contoh kalimat pertama, kata \textbf{bisa} dapat diterjemahkan menjadi \textit{can} atau \textit{could}. Berbeda dengan kalimat kedua yang mana kata \textbf{bisa} diterjemahkan menjadi \textit{venom}. Mengetahui makna kata (\textit{sense}) yang tepat dapat mempermudah proses penerjemahan otomatis untuk sebuah kata.

%-----------------------------------------------------------------------------%
\section{Perumusan Masalah}
%-----------------------------------------------------------------------------%
Beberapa pertanyaan yang menjadi rumusan masalah dalam penelitian ini yaitu:
\begin{enumerate}
	\item Bagaimana cara menerapkan pemindahan makna (\textit{sense transfer}) dari korpus paralel bahasa Inggris - Indonesia?
	\item Seberapa baik performa \textit{WSD} yang dibangun untuk bahasa Indonesia tersebut?
\end{enumerate}

%-----------------------------------------------------------------------------%
\section{Tujuan dan Manfaat Penelitian}
%-----------------------------------------------------------------------------%
Tujuan dari penelitian yang dilakukan adalah memberikan tambahan inventaris berupa \textit{sense} dari kata-kata bahasa Indonesia untuk wordnet Bahasa Indonesia dan menghitung seberapa baik performa \textit{wsd system} bahasa Indonesia yang dibuat.

%-----------------------------------------------------------------------------%
\section{Ruang Lingkup Penelitian}
Penelitian berfokus pada pemindahan \textit{sense} dari korpus paralel berbahasa Inggris ke Indonesia, dan melakukan \textit{WSD task} pada hasil pemindahan makna kata tersebut.

\textit{Word sense disambiguation} yang dilakukan pada penelitian ini hanya pada tingkatan \textit{coarse-grained} \textit{wsd}.

%-----------------------------------------------------------------------------%
\section{Metodologi Penelitian}
%-----------------------------------------------------------------------------%
Ada lima tahapan yang dilakukan pada penelitian ini. Penjelasan dari tiap tahapan adalah sebagai berikut.
\begin{enumerate}
	\item Studi Literatur \\
	Tahap ini berfokus pada pencarian informasi mengenai \textit{WSD system} baik secara umum maupun teknik yang digunakan, dan juga \textit{task} lain yang berkaitan dengan \textit{WSD} seperti \textit{word sense induction} (\textit{WSI}).
	\item Perumusan Masalah \\
	Masalah-masalah yang ada dalam penelitian nantinya dianalisis penyelesaiannya pada tahap ini.
	\item Pemindahan Sense Korpus English ke Indonesia\\
	Proses yang melibatkan \textit{sense tagging} pada korpus bahasa inggris, \textit{word alignment} korpus Indonesia-English, dan pemindahan \textit{sense}.
	\item Implementasi Sistem WSD  \\
	Tahap ini akan membahas perihal implementasi dari sistem WSD yang dibangun untuk disambiguasi pada korpus Indonesia dengan kata-kata yang ditentukan.
	\item Analisis dan Kesimpulan \\
	Hasil eksperimen dianalisis untuk melihat seberapa baik sistem wsd yang telah dibangun.
\end{enumerate}


%-----------------------------------------------------------------------------%
\section{Sistematika Penulisan}
%-----------------------------------------------------------------------------%
Sistematika penulisan yang ada dalam laporan penelitian ini sebagai berikut:
\begin{itemize}
	
	\item Bab 1 \babSatu \\
	Bab ini akan menjelaskan mengenai latar belakang, perumusan masalah, tujuan penelitian, tahapan penelitian, ruang lingkup, metodologi, dan sistematika penulisan dari penelitian ini.
	
	\item Bab 2 \babDua \\
	Bab ini akan menjelaskan mengenai konsep dan teori yang relevan dari hasil studi literatur yang telah dilakukan. Teori-teori yang dijelaskan meliputi \textit{Word sense disambiguation}, \textit{Word sense induction}, dan beberapa hal lain yang dibutuhkan pada penelitian.
	
	\item Bab 3 \babTiga \\
	Bab ini akan membahas perihal pelaksanaan dari proses \textit{tagging sense} pada korpus English dan \textit{word alignment} pada kedua korpus (Indonesia - English) beserta evaluasinya.
	
	\item Bab 4 \babEmpat \\
	Pada bab ini akan dijelaskan mengenai implementasi dari sistem WSD yang dibangun untuk melakukan disambiguasi pada kata-kata yang telah ditentukan.
	
	\item Bab 5 \babLima \\
	Pada bab ini, dijelaskan mengenai hasil penelitian beserta evaluasi dan analisis dari hasil tersebut. 
	
	\item Bab 6 \babEnam \\
	Kesimpulan dan saran dari hasil dan pelaksanaan penelitian akan dijelaskan pada bab ini.
	
\end{itemize}
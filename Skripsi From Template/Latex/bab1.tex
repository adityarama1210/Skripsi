%!TEX root = skripsi.tex
%-----------------------------------------------------------------------------%
\chapter{\babSatu}
%-----------------------------------------------------------------------------%
Bab ini membahas mengenai latar belakang penelitian, perumusan masalah, tujuan dan manfaat penelitian, ruang lingkup penelitian, metodologi penelitian, serta sistematika penulisan.

%-----------------------------------------------------------------------------%
\section{Latar Belakang}
%-----------------------------------------------------------------------------%	

\textit{Word Sense Disambiguation} (WSD) merupakan salah satu tugas untuk menentukan makna terbaik dari sebuah kata. Sebuah kata sendiri dapat memiliki beberapa makna dan bergantung pada konteks dimana kata tersebut muncul. Penentuan makna kata yang paling tepat ini secara tidak langsung dapat membantu beberapa \textit{task} \textit{Natural Language Processing}(NLP) ataupun \textit{Information Retrieval} (IR) lainnya seperti misalnya \textit{machine translation} (MT). Salah satu contoh dimana WSD dapat membantu dalam sistem IR adalah pada \textit{task question answering} berikut.

\begin{lstlisting}[backgroundcolor = \color{white}]
Q: Apakah orang Indonesia sarapan apel di pagi hari?
A: Ya, orang indonesia sering melakukan apel di pagi hari setelah mereka sarapan dan pergi ke kantor 
\end{lstlisting}
Pada contoh kasus \textit{question answering} diatas, dapat terjadi jawaban dari \textit{query} yang diberikan tidak cocok dengan pertanyaan karena sistem tidak mengerti bahwa "apel" yang dimaksud pada kalimat pertanyaan adalah apel (buah), bukan merupakan apel (upacara). Sistem WSD yang dapat melakukan disambiguasi makna kata dapat membantu sistem QA diatas untuk dapat mengetahui bahwa "apel" dalam pertanyaan memiliki makna sebagai buah.

Membangun sistem WSD biasanya dilakukan dengan beberapa pendekatan \textit{machine learning} seperti \textit{supervised learning}, \textit{semi supervised}, ataupun \textit{unsupervised}. Pendekatan \textit{supervised machine learning} yang dapat digunakan untuk membangun sistem WSD membutuhkan data yang tidak sedikit. Data yang dibutuhkan untuk sistem ini dapat berupa \textit{sense-tagged corpus} dimana isinya adalah kata-kata yang sudah mempunyai kelas makna kata yang tepat. Kebutuhan akan data yang relatif besar tersebut merupakan kendala yang ada pada bahasa-bahasa tertentu. Bahasa Inggris sebagai salah satu bahasa internasional mempunyai data yang cukup banyak untuk membangun sistem dengan \textit{supervised learning}. Namun demikian, bahasa Indonesia sendiri termasuk dalam \textit{under resource language} dimana data yang dapat dimanfaatkan untuk sistem WSD masih terbatas. Belum adanya data seperti \textit{sense-tagged corpus} untuk membangun sistem WSD bahasa Indonesia merupakan salah satu permasalahan yang dihadapi jika dibandingkan dengan bahasa Inggris.

Makna kata yang diberikan oleh sistem WSD pada kata-kata pada umumnya berasal dari definisi pada Wordnet. Namun demikian, membangun Wordnet secara manual untuk memenuhi kebutuhannya sebagai inventaris makna kata membutuhkan waktu dan dana yang relatif tidak sedikit. Terdapat beberapa Wordnet bahasa Indonesia yang dapat digunakan, diantaranya adalah Wordnet Bahasa (bahasa.cs.ui.ac.id) dan Wordnet Bahasa yang berasal dari proyek Nanyang Technological University (NTU).  Wordnet dari Bahasa Fasilkom UI masih memiliki sedikit \textit{sets of synonyms} (synsets) yang mana berarti banyak kata-kata yang tidak tersedia di dalamnya. Wordnet bahasa dari proyek NTU sendiri masih mengandung makna kata yang tidak sesuai. Oleh karena masalah tersebut, pada penelitian ini Wordnet yang digunakan adalah Wordnet bahasa Inggris buatan Princeton yang sudah sering digunakan untuk \textit{event} Semantic Evaluation atau Sense Evaluation. Penggunaan makna kata dari Wordnet bahasa Inggris tersebut nantinya akan dipindahkan ke kata dalam bahasa Indonesia yang bersesuaian dengan memanfaatkan sifat paralel dari korpus Identic hasil penelitian \citep{larasati2012identic}.

Perbedaan penelitian ini dengan \textit{cross lingual WSD Bahasa Indonesia} \citep{septiantri2013wsd} adalah bahwa penelitian ini lebih berfokus pada menghasilkan \textit{sense tagged corpus} agar dapat digunakan sebagai \textit{training} data WSD Bahasa Indonesia. Pendekatan \textit{cross lingual sense transfering} dengan bahasa Inggris sebagai pasangan korpus diharapkan dapat membangun \textit{sense tagged corpus} yang masih kurang pada bahasa Indonesia. 
%-----------------------------------------------------------------------------%
\section{Perumusan Masalah}
%-----------------------------------------------------------------------------%
Beberapa pertanyaan yang menjadi rumusan masalah dalam penelitian ini yaitu:
\begin{enumerate}
	\item Bagaimana cara membangun \textit{sense tagged corpus} Bahasa Indonesia dengan pendekatan \textit{cross lingual} dari paralel korpus?
	\item Seberapa baik performa WSD dari \textit{sense tagged corpus} pada tahap pertama?
\end{enumerate}

%-----------------------------------------------------------------------------%
\section{Tujuan dan Manfaat Penelitian}
%-----------------------------------------------------------------------------%
Tujuan dari penelitian ini adalah menghasilkan \textit{sense tagged corpus} dalam bahasa Indonesia, dan menghitung seberapa baik performa sistem WSD bahasa Indonesia yang dibuat.
%-----------------------------------------------------------------------------%
\section{Ruang Lingkup Penelitian}
Penelitian berfokus pada pemindahan \textit{sense} dari korpus paralel berbahasa Inggris ke Indonesia, dan melakukan \textit{WSD task} dari hasil \textit{sense transfering} tersebut.
%-----------------------------------------------------------------------------%
\section{Metodologi Penelitian}
%-----------------------------------------------------------------------------%
Ada lima tahapan yang dilakukan pada penelitian ini. Penjelasan dari tiap tahapan adalah sebagai berikut.
\begin{enumerate}
	\item Studi Literatur \\
	Tahap ini berfokus pada pencarian informasi mengenai \textit{WSD system} baik secara umum maupun teknik yang digunakan, dan juga \textit{task} lain yang berkaitan dengan \textit{WSD} seperti \textit{word sense induction} (\textit{WSI}).
	\item Perumusan Masalah \\
	Masalah-masalah yang ada dalam penelitian nantinya dianalisis penyelesaiannya pada tahap ini.
	\item Rancangan Penelitian\\
	Proses yang melibatkan seluruh penelitian untuk menyelesaikan permasalahan yang ada.
	\item Implementasi\\
	Tahap ini merupakan implementasi dari rancangan yang sudah dibuat untuk memecahkan permasalahan yang ada.
	\item Analisis dan Kesimpulan \\
	Hasil percobaan dianalisis untuk mendapatkan gambaran seberapa baik performa dari sistem yang dibuat.
\end{enumerate}


%-----------------------------------------------------------------------------%
\section{Sistematika Penulisan}
%-----------------------------------------------------------------------------%
Sistematika penulisan yang ada dalam laporan penelitian ini sebagai berikut:
\begin{itemize}
	
	\item Bab 1 \babSatu \\
	Bab ini akan menjelaskan mengenai latar belakang, perumusan masalah, tujuan penelitian, tahapan penelitian, ruang lingkup, metodologi, dan sistematika penulisan dari penelitian ini.
	
	\item Bab 2 \babDua \\
	Bab ini akan menjelaskan mengenai konsep dan teori yang relevan dari hasil studi literatur yang telah dilakukan. Teori-teori yang dijelaskan meliputi \textit{Word sense disambiguation}, \textit{Word sense induction}, dan beberapa hal lain yang dibutuhkan pada penelitian.
	
	\item Bab 3 \babTiga \\
	Bab ini akan membahas perihal rancangan dari proses penelitian yang meliputi \textit{sense tagging} korpus Inggris, \textit{word alignment} Indonesia-Inggris, \textit{sense transfering}, dan sistem WSD bahasa Indonesia.
	
	\item Bab 4 \babEmpat \\
	Pada bab ini akan menjelaskan mengenai implementasi dari rancangan sistem yang dibuat.
	
	\item Bab 5 \babLima \\
	Pada bab ini, dijelaskan mengenai hasil penelitian beserta evaluasi dan analisisnya. 
	
	\item Bab 6 \babEnam \\
	Kesimpulan dan saran dari hasil dan pelaksanaan penelitian akan dijelaskan pada bab ini.
	
\end{itemize}
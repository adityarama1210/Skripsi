%!TEX root = skripsi.tex
%-----------------------------------------------------------------------------%
\chapter{\babSatu}
%-----------------------------------------------------------------------------%
Bab ini membahas mengenai latar belakang penelitian, perumusan masalah, tujuan dan manfaat penelitian, ruang lingkup penelitian, metodologi penelitian, serta sistematika penulisan.

%-----------------------------------------------------------------------------%
\section{Latar Belakang}
%-----------------------------------------------------------------------------%	

\textit{Task word sense disambiguation} merupakan pekerjaan yang mempunyai tujuan utama untuk mengenali \textit{sense}(makna) yang tepat dari sebuah kata. Sebagai manusia, pekerjaan ini termasuk relatif mudah karena kita dapat secara mengenal \textit{sense} dari sebuah kata berdasarkan makna dari konteks yang ada di sekelilingnya. Komputer pada dasarnya hanya dapat mengenali kata sebagai sebuah representasi data, sehingga perlu adanya \textit{modeling} agar komputer dapat mengenal dan membedakan makna antara satu kata dengan kata lainya. Sebuah contoh sederhana dari penentuan makna yang tepat dari sebuah kata dapat dilihat pada kalimat berikut.

(S1) : Setiap orang \textbf{bisa} melakukan tindak kriminal.

(S2) : Racun dari \textbf{bisa} ular itu sudah menyebar di tubuhnya.

(S3) : Ani sangat senang memakan \textbf{cokelat} pemberian Budi.

(S4) : Budi mempunyai sofa \textbf{cokelat} yang sangat mahal

Kata yang memiliki makna ganda (polisemi) pada kalimat satu dan kalimat dua adalah "bisa". kata "bisa" pada kedua kalimat tersebut memiliki makna yang berbeda, dimana kata "bisa" dalam kalimat satu menunjukan "kemampuan seseorang untuk melakukan sesuatu", dan kata "bisa" kalimat dua merujuk pada makna "racun yang berasal dari gigitan ular". Pada contoh permasalahan kalimat satu dan dua, penyelesaian relatif cukup mudah jika dilakukan dengan pendekatan \textit{POS Tagging}. Penggunaan bantuan \textit{task POS Tagging} tersebut dapat langsung membedakan bahwa "bisa" pada kalimat satu adalah kata kerja, sementara "bisa" kalimat kedua adalah objek. Kalimat ketiga dan keempat memberikan contoh lain dari makna yang berbeda pada kata yang sama berdasarkan konteks yang berbeda. Cokelat pada kalimat pertama bermakna makanan manis olahan dari hasil pohon cokelat sementara cokelat kalimat keempat merupakan warna. Kedua kata 'cokelat' tersebut sama-sama mempunyai \textit{POS Tag} berupa \textit{noun}, namun demikian makna yang terkandungnya berbeda.

Pengartian makna untuk sebuah kata sendiri merupakan masalah yang kompleks dan tidak mudah. Sebagai manusia, terdapat berbagai macam cara ataupun sudut pandang untuk merepresentasikan kata dari makna sebuah kalimat/paragraf. Sebagai contoh pisau pada bahasan sebelumnya, kita dapat membedakan makna pisau berdasarkan kegunaannya, manfaatnya, dampaknya, dan lain-lain.

Penggunaan dari \textit{word sense disambiguation} ini dapat bermanfaat untuk \textit{sub task} pada \textit{machine translation}, dimana satu buah kata dapat diterjemahkan menjadi beberapa kemungkinan kata yang berbeda berdasarkan makna yang dikandungnya. Mengacu pada contoh kalimat pertama, kata \textbf{bisa} dapat diterjemahkan menjadi\textit{can} atau \textit{could}. Berbeda dengan kalimat kedua yang mana kata \textbf{bisa} diterjemahkan menjadi \textit{venom}. Mengetahui makna kata (\textit{sense}) yang tepat dapat mempermudah proses penerjemahan otomatis untuk \textit{unit} kata.

%-----------------------------------------------------------------------------%
\section{Perumusan Masalah}
%-----------------------------------------------------------------------------%
Beberapa pertanyaan yang menjadi rumusan masalah dalam penelitian ini yaitu:
\begin{enumerate}
	\item Bagaimana cara menerapkan pemindahan makna (\textit{sense transfer}) dari korpus paralel bahasa Inggris - Indonesia?
	\item Seberapa baik performa \textit{WSD} yang dibangun untuk bahasa Indonesia tersebut?
\end{enumerate}

%-----------------------------------------------------------------------------%
\section{Tujuan dan Manfaat Penelitian}
%-----------------------------------------------------------------------------%
Tujuan dari penelitian yang dilakukan adalah mencari fitur yang optimal untuk \textit{wsd system} bahasa Indonesia dan memberikan tambahan \textit{sense} kata-kata dalam bahasa Indonesia yang dapat dipindahkan dari \textit{sense} kata bahasa Inggris.

%-----------------------------------------------------------------------------%
\section{Ruang Lingkup Penelitian}
Penelitian ini berfokus pada pembuatan korpus \textit{Textual Entailment} Bahasa Indonesia, sehingga teks yang dikenali dengan baik dalam penelitian ini adalah teks berbahasa Indonesia. Korpus dikembangkan menggunakan metode \textit{semi-supervised learning}, yaitu Co-training. Sumber data untuk pembuatan korpus adalah Wikipedia \textit{revision history} berbahasa Indonesia.

Fenomena \textit{Textual Entailment} memiliki beberapa tingkatan, yaitu leksikal, sintaktik, leksikal-sintaktik, \textit{discourse}, dan \textit{reasoning}. Wikipedia \textit{revision history} mungkin tidak akan menangkap semua fenomena \textit{Textual Entailment} tersebut. Sehingga, cakupan fenomena \textit{Textual Entailment} yang terdapat pada hasil korpus penelitian ini terbatas pada tingkatan tertentu.

%-----------------------------------------------------------------------------%
\section{Metodologi Penelitian}
%-----------------------------------------------------------------------------%
Ada lima tahapan yang dilakukan pada penelitian ini. Penjelasan dari tiap tahapan adalah sebagai berikut.
\begin{enumerate}
	\item Studi Literatur \\
	Pada tahap ini, literatur-literatur yang berkaitan dengan penelitian \textit{Textual Entailment} dipelajari terlebih dahulu agar \textit{state of the art} dari penelitian ini menjadi lebih jelas.
	\item Perumusan Masalah \\
	Setelah mempelajari berbagai literatur, masalah yang akan diselesaikan dari penelitian ini dirumuskan agar hasil yang diharapkan di akhir penelitian lebih tergambar jelas. 
	\item Perancangan Pengembangan Korpus\\
	Proses-proses yang akan dilakukan dalam penelitian pengembangan korpus \textit{Textual Entailment} terlebih dahulu dirancang, seperti pengumpulan dan pengolahan data Wikipedia \textit{revision history}, hingga percobaan Co-training.
	\item Implementasi  \\
	Tahap ini adalah inti dari penelitian. Implementasi dilakukan untuk mencari tahu jawaban atas rumusan masalah. 
	\item Analisis dan Kesimpulan \\
	Hasil eksperimen dianalisis lebih dalam pada tahapan ini. Kemudian disimpulkan sebagai jawaban dari rumusan masalah.
\end{enumerate}


%-----------------------------------------------------------------------------%
\section{Sistematika Penulisan}
%-----------------------------------------------------------------------------%
Sistematika penulisan yang ada dalam laporan penelitian ini sebagai berikut:
\begin{itemize}
	
	\item Bab 1 \babSatu \\
	Pada bab ini, dijelaskan mengenai latar belakang yang menjadi dasar dari penulisan. Selain itu, bab ini juga menjelaskan mengenai perumusan masalah, tujuan penelitian, tahapan penelitian, ruang lingkup penelitian, metodologi penelitian, serta sistematika penulisan dari penelitian ilmiah ini.
	
	\item Bab 2 \babDua \\
	Pada bab ini, dijelaskan mengenai teori-teori yang relevan berdasarkan hasil studi literatur yang telah dilakukan. Studi literatur ini menjelaskan tentang \textit{Textual Entailment}, Wikipedia \textit{revision history}, algoritme Co-training dan hal-hal mendasar lainnya yang dibutuhkan dalam penelitian.
	
	\item Bab 3 \babTiga \\
	Rancangan pengembangan korpus \textit{Textual Entailment} Bahasa Indonesia akan dibahas pada bab ini. Salah satu pembahasan dalam bab ini adalah bagaimana cara mengubah Wikipedia \textit{revision history} menjadi data kandidat isi korpus \textit{Textual Entailment} dan bagaimana menerapkan Co-training pada permasalahan tersebut.
	
	\item Bab 4 \babEmpat \\
	Bab ini menjelaskan mengenai implementasi penelitian, mulai dari tahap pengolahan data Wikipedia \textit{revision history} hingga percobaan Co-training.
	
	\item Bab 5 \babLima \\
	Pada bab ini, dijelaskan mengenai hasil penelitian beserta evaluasi dan analisis dari hasil tersebut. 
	
	\item Bab 6 \babEnam \\
	Kesimpulan penelitian dan saran terkait penelitian dijelaskan pada bab ini. Kesimpulan akan memberikan jawaban atas pertanyaan dari rumusan masalah penelitian. Saran diberikan agar penelitian selanjutnya dapat memperbaiki kekurangan-kekurangan dalam penelitian ini.
	
\end{itemize}
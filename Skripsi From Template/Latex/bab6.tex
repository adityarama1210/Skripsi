%!TEX root = skripsi.tex
%-----------------------------------------------------------------------------%
\chapter{\babEnam}
%-----------------------------------------------------------------------------%

%-----------------------------------------------------------------------------%
\section{Kesimpulan}
%-----------------------------------------------------------------------------%

Data dan \textit{resource} berupa \textit{sense tagged corpus} yang terbatas pada bahasa Indonesia merupakan penghambat dari penelitian WSD di Bahasa Indonesia. Untuk mengatasi masalah tersebut, salah satu pendekatan WSI berupa \textit{cross lingual} dapat dimanfaatkan untuk \textit{transfering knowledge} dari salah satu bahasa dengan data yang lebih banyak yaitu bahasa Inggris.

Konsep pendekatan \textit{cross lingual sense transfering} ini dapat menghasilkan \textit{sense tagged corpus} bahasa Indonesia dengan memindahkan makna kata dari korpus bahasa Inggris ke kata-kata yang menjadi pasangannya (\textit{translation} dari kata tersebut) di dalam bahasa indonesia. Metode yang digunakan untuk membangun \textit{sense tagged corpus} tersebut meliputi \textit{tagging} pada korpus Bahasa Inggris dengan menggunakan \textit{tool} IMS \citep{zhong2010makes}, melakukan \textit{alignment} kata pada korpus identik (Inggris-Indonesia) dengan Giza++ \citep{och03:asc}, dan \textit{sense transfering} untuk memindahkan makna kata tersebut. \textit{Sense tagged corpus} Bahasa Indonesia  yang merupakan hasil dari proses tersebut memiliki kualitas cukup baik walaupun masih dapat ditingkatkan lagi.

Pada penelitian yang dilakukan, performa sistem WSD yang dibuat dengan \textit{classifier} SVM dan empat skenario fitur, memiliki performa yang lebih baik dari baseline pada kebanyakan kasus. Fitur dengan persentase terbaik adalah gabungan dari fitur \textit{bag of words} dan POS Tag yang mengungguli baseline pada 80\% dari sepuluh \textit{sample} percobaan.
%-----------------------------------------------------------------------------%
\section{Saran}
%-----------------------------------------------------------------------------%
Setelah melakukan percobaan dan melakukan analisis hasil dari penelitian ini, terdapat beberapa saran untuk penelitian selanjutnya diantaranya sebagai berikut.

\begin{enumerate}
	\item Penggunaan kualitas dari korpus paralel dapat mempengaruhi seberapa baik hasil \textit{sense transfering} yang dilakukan. Pada korpus identik yang penelitian ini gunakan dengan jumlah kalimat sebanyak 88.919 buah, terdapat beberapa kalimat yang terpotong maupun \textit{comparable} (tidak sepenuhnya paralel), sehingga rawan menimbulkan kesalahan \textit{alignment}.
	\item Fitur yang digunakan dalam sistem WSD bahasa Indonesia masih dapat ditambahkan dengan berbagai fitur lainnya seperti Collocation, \textit{text rank}, \textit{dependancy parser}, dan lain-lain.
	\item \textit{Classifier} pada sistem WSD juga dapat diganti dengan yang lain seperti misalnya Naive Bayes, Multilayer Perceptron, dan lain-lain. Arsitektur juga dapat dicoba untuk menggunakan pendekatan \textit{deep learning} pada penelitian selanjutnya.
	\item Proses \textit{word alignment} sebaiknya dilakukan dengan \textit{tool} SMT yang lebih baru, salah satunya adalah \textit{berkeley aligner}.
\end{enumerate}
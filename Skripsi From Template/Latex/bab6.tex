%!TEX root = skripsi.tex
%-----------------------------------------------------------------------------%
\chapter{\babEnam}
%-----------------------------------------------------------------------------%

%-----------------------------------------------------------------------------%
\section{Kesimpulan}
%-----------------------------------------------------------------------------%

Data dan \textit{resource} berupa \textit{sense tagged corpus} yang terbatas pada bahasa Indonesia merupakan penghambat dari penelitian WSD di bahasa Indonesia. Untuk mengatasi masalah tersebut, salah satu pendekatan WSI berupa \textit{cross language} dapat dimanfaatkan untuk \textit{transfering knowledge} dari salah satu bahasa dengan data yang mumpuni yaitu bahasa Inggris. Konsep pendekatan ini dapat menghasilkan \textit{sense tagged corpus} bahasa Indonesia dengan memindahkan makna kata dari korpus bahasa Inggris ke kata-kata yang menjadi pasangannya (\textit{translation} dari kata tersebut) di dalam bahasa indonesia.
%-----------------------------------------------------------------------------%
\section{Saran}
%-----------------------------------------------------------------------------%

%!TEX root = skripsi.tex
%-----------------------------------------------------------------------------%
\chapter{\babDua}
%-----------------------------------------------------------------------------%
Bab ini membahas mengenai studi literatur yang digunakan selama penelitian. Studi literatur ini menjelaskan tentang hal-hal mendasar yang dibutuhkan dalam penelitian.

%-----------------------------------------------------------------------------%
\section{Word Sense Disambiguation}
%-----------------------------------------------------------------------------%
\textit{Word Sense Disambiguation} merupakan salah satu penelitian di bidang NLP yang bertujuan untuk menentukan makna yang paling tepat dari suatu kata berdasarkan konteks kata tersebut ditemukan. Sebagaimana kata dalam suatu bahasa bisa memiliki makna lebih dari satu (polisemi), \textit{task} ini akan menentukan makna kata mana yang paling tepat.

Penentuan makna kata yang tepat oleh sistem WSD ditentukan berdasarkan konteks dari kata tersebut berada. Walaupun satu kata dapat memiliki beberapa makna, terdapat kecil kemungkinan bahwa kata yang sama digunakan dalam satu \textit{discourse} untuk menyatakan makna yang berbeda sebagaimana "one sense per discourse" \citep{gale1992one}

%-----------------------------------------------------------------------------%
\section{Word Sense Induction}
%-----------------------------------------------------------------------------%	
%-----------------------------------------------------------------------------%	
\textit{Word Sense Induction} (WSI) adalah sebuah \textit{task} yang mempunyai fungsi utama untuk mendapatkan makna kata dari sebuah korpus atau teks yang belum dianotasi secara otomatis. WSI dapat dilakukan jika penelitian WSD yang ingin dilakukan tidak mempunyai cukup \textit{resource} seperti misalnya Wordnet yang memadai. Terdapat berbagai macam pendekatan dalam melakukan WSI, diantaranya adalah dengan melakukan \textit{clustering} kata \citep{denkowski2009survey}, ataupun menggunakan pendekatan \textit{cross language}.
	
	\subsection{Pendekatan \textit{Clustering}}
	Dua kata dianggap dekat secara semantik jika memiliki \textit{co-occurrence} dengan kata-kata tetangganya yang sama \citep{nasiruddin2013state}. Konsep tersebut mendasari cara WSI mendapatkan \textit{sense} kata secara implisit berdasarkan hasil \textit{cluster} yang terbentuk dari data atau teks mentah (teks yang tidak dianotasi). 
	
	\subsection{Pendekatan \textit{Cross Language}}
	Selain pendekatan \textit{clustering}, WSI juga dapat memanfaatkan fitur dimana satu kata dari suatu bahasa, dapat diterjemahkan menjadi beberapa kata di bahasa lain. Contoh kasus tersebut dapat dilihat pada kata "halaman" berikut:



	(K1-Indonesia): Aku membaca 10 \textbf{halaman} buku Harry Potter
	
	(K1-English): I read 10 \textbf{pages} of Harry Potter book
	
	(K2-Indonesia): Ani tinggal di rumah dengan \textbf{halaman} yang sangat luas
	
	(K2-English): Ani lives in a house with very large \textbf{yard}
	
	
	
	Berdasarkan kedua pasangan kalimat tersebut, kata \textbf{halaman} dalam bahasa Indonesia dapat diterjemahkan menjadi dua buah kata dalam bahasa Inggris, yaitu \textit{page} ataupun \textit{yard}. Hal ini menunjukan bahwa terjemahan dari suatu kata bergantung pada makna yang dikandung kata tersebut.

%-----------------------------------------------------------------------------%
\section{Evaluasi}
%-----------------------------------------------------------------------------%

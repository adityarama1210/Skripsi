%!TEX root = skripsi.tex
%-----------------------------------------------------------------------------%
\chapter{\babLima}
%-----------------------------------------------------------------------------%
Bab ini menjelaskan mengenai hasil yang didapatkan dari eksperimen, serta evaluasi dan analisis terkait hasil tersebut.

%-----------------------------------------------------------------------------%
\section{Korpus Identik}
%-----------------------------------------------------------------------------%
Korpus identik berisi pasangan-pasangan kalimat Indonesia-Inggris sebanyak 88.919 kalimat untuk masing-masing bahasa.
%-----------------------------------------------------------------------------%

%-----------------------------------------------------------------------------%
\section{Pembuatan \textit{Sense Tagged Corpus} Bahasa Inggris}
Tabel \ref{table:sense-tagged-corpus} menunjukan jumlah token (kata) pada korpus bahasa Inggris dan yang diberikan \textit{tag sense} oleh IMS

\begin{table}
	\centering
	\caption{Jumlah \textit{instance} korpus bahasa Inggris}
	\label{table:sense-tagged-corpus}
	\begin{tabular}{|p{0.7cm}|p{4cm}|p{4cm}|}
		\hline
		No & Tipe & Jumlah
		\\ \hline
		1    & 
		Token (kata)   & 
		1.801.484
		\\ \hline
		2    & 
		Kata yang diberikan \textit{tag} oleh IMS     & 
		1.024.797 
		\\ \hline
	\end{tabular}
\end{table}

Berdasarkan proses pembuatan dan hasil dari \textit{sense tagged corpus} tersebut, dapat dilihat bahwa tidak semua kata diberikan \textit{sense} oleh IMS. Kata-kata sapaan seperti "I", "you", dan kata \textit{articles} yaitu "a", "the", "an". Selain itu, kata yang tidak terdapat pada model juga tidak diberi \textit{tag} (dilewati) oleh IMS. Tingkat kebenaran dari \textit{sense tag} yang diberikan bergantung dari model yang digunakan pada penelitian. Terdapat banyak kasus-kasus dimana pemberian \textit{sense} yang dilakukan adalah benar seperti misalnya pada kata "\textit{visitor}" yang diberikan \textit{tag} dengan \textit{sense key} 1:18:00::, yang mana berdasarkan wordnet Princeton "visitor\%1:18:00::" memiliki arti sebagai "\textit{someone who visits}". Contoh lain dari kata yang diberikan \textit{tag} dengan benar adalah "\textit{company}" pada konteks potongan kalimat "\textit{Plantation company PT ...}". Kata "\textit{company}" tersebut diberikan tag "company\%1:14:01::" yang berdasarkan wordnet Princeton memiliki makna "\textit{an institution created to conduct business}". Namun demikian, terjadinya kesalahan pemberian \textit{tag} pada kata terjadi pada kasus-kasus seperti: 

\begin{enumerate}
	\item Sebuah entitas diberikan \textit{tag} dimana entitas tersebut dianggap kata biasa. Contohnya adalah kata "Scotland Yard" dimana "Yard" pada kata tersebut diberikan \textit{tag} yang diartikan sebagai "\textit{a unit of length equal to 3 feet}". Hal ini menunjukan bahwa \textit{tool} belum dapat membedakan antara entitas yang memang tidak perlu diberikan \textit{tag} dan kata biasa (walaupun kata tersebut sudah memiliki huruf kapital).
	\item Kesalahan \textit{tag} dikarenakan \textit{training data} yang digunakan oleh model. Pada potongan kalimat "\textit{... FASB rule will cover such financial instruments as interest rate swaps financial ...}, kata "\textit{interest}" diberikan tag dengan makan "a sense of concern with and curiosity about someone or something". Berdasarkan konteks kalimat tersebut, dapat diketahui bahwa makna yang seharusnya didapat untuk kata "\textit{interest}" diatas ialah "bunga bank". Hal ini sepertinya terjadi karena data yang digunakan untuk \textit{training} model IMS memiliki ketidakseimbangan data untuk model kata "\textit{interest}" sehingga \textit{tag} yang diberikan lebih cenderung kepada "ketertarikan".
	\item Pemberian \textit{tag} pada \textit{multi word} token seperti "\textit{make up}" masih diberikan pada setiap kata. Berdasarkan percobaan untuk \textit{tagging} pada kata tersebut, kata "\textit{make}" dan "\textit{up}" masing-masing diberikan tag yang berbeda. Hal ini terjadi karena IMS mengolah kata demi kata dengan proses tokenisasi \textit{by default} menggunakan spasi. Setelah dilakukan pemeriksaan pada kata-kata yang terdapat pada model, kata \textit{make up} ternyata disimpan sebagai "make\_up". Berdasarkan pemeriksaan tersebut, diperlukan adanya \textit{pre-processing} terlebih dahulu untuk mengganti \textit{separator} kata multiword yang umumnya menggunakan spasi dengan "\_" agar IMS dapat memberikan \textit{tag multi word} tersebut dengan benar. Selain \textit{pre-processing},  IMS juga dapat melakukan \textit{tagging} dengan input dalam format XML. Bentuk kata-kata dan kalimat dalam format XML tersebut biasanya memiliki multi word yang sudah dijadikan satu token sehingga mempermudah penyelesaian masalah tersebut.
\end{enumerate}
%-----------------------------------------------------------------------------%

%-----------------------------------------------------------------------------%
\section{Evaluasi \textit{Word Alignment}}

Hasil dari proses \textit{word alignment} yang dilakukan Giza dibandingkan dengan hasil \textit{alignment} yang dibuat oleh dua orang anotator. Jumlah yang akan dibandingkan adalah 200 buah pasangan data yang didapat dengan \textit{random sampling}. Indikator performa dari perbandingan tersebut adalah nilai dari \textit{precision} dan \textit{recall}.

%-----------------------------------------------------------------------------%
\section{\textit{Sense Transfering}}

Proses \textit{transfer} makna kata dari bahasa Inggris ke bahasa Indonesia yang dilakukan sangat bergantung dari hasil \textit{alignment} kata pada proses sebelumnya. Untuk sebagian besar kata yang memiliki pasangan kata yang benar, proses \textit{transfer} dapat menghasilkan makna yang benar juga. Hal tersebut didukung jika \textit{sense tagged word} pada korpus bahasa Inggris juga benar). Terdapat beberapa kata yang dipilih sebagai \textit{sampling} untuk mengevaluasi hasil \textit{sense transfering}. Kelompok ini dibagi menjadi:

\begin{enumerate}
	\item Jumlah Kelas
	\begin{enumerate}
		\item 3-5 kelas kata
		\item lebih dari 5 kelas kata
	\end{enumerate}
	\item sebaran jumlah \textit{instance} dalam kelasnya
	\begin{enumerate}
		\item \textit{balance}
		\item \textit{imbalance}
	\end{enumerate}
	\item Bentuk morfologi dari kata tersebut
	\begin{enumerate}
		\item Lemma
		\item Berimbuhan baik itu infleksional ataupun \textit{derivative}
	\end{enumerate}
\end{enumerate}

Pada jumlah kelas sebanyak 3-5 kelas kata (\textit{sense key}), \textit{target word} yang diambil adalah "memecahkan". Kata tersebut memiliki 4 buah kelas total dengan \textit{sense key} yang didapat yaitu 'solve\%2:31:00::','resolve\%2:31:01::', 'break\%2:30:03::', dan 'split\%2:38:00::'. Kata "menolak" mewakili kelas kata sebanyak 10 buah yang diantaranya mengandung kelas 'refuse\%2:32:00::', 'reject\%2:40:00::', 'decline\%2:32:00::', dan beberapa kelas lainnya. Tabel \ref{table:number-classes-sense-transfering-evaluation} menunjukan contoh beberapa kata tersebut dalam konteks yang berkaitan.

\begin{table}
	\centering
	\caption{Evaluasi \textit{Sense Transfering} Berdasarkan Jumlah Kelompok Kata}
	\label{table:number-classes-sense-transfering-evaluation}
	\begin{tabular}{|p{4cm}|p{4cm}|p{4cm}|}
		\hline
		\textit{Sense Key} & Makna & Kalimat
		\\ \hline
		solve\%2:31:00::  & 
		\textit{find the solution to (a problem or question) or understand the meaning of}   & 
		salah satu cara untuk \textbf{memecahkan} persoalan yang pelik...
		\\ \hline
		resolve\%2:31:01:: & 
		\textit{bring to an end / settle conclusively}   & 
		evolusionis masih belum bisa \textbf{memecahkan} permasalahan darwin...
		\\ \hline
		break\%2:30:03:: & 
		\textit{terminate}   & 
		...base mereka \textbf{memecahkan} rekor untuk...
		\\ \hline
	\end{tabular}
\end{table}

%-----------------------------------------------------------------------------%
\section{Sistem WSD}

Untuk melihat seberapa baik performa sistem WSD dengan menggunakan \textit{sense tagged corpus} hasil dari penelitian, terdapat kata-kata yang dipilih secara manual sebagai \textit{target word} yang akan dievaluasi berdasarkan nilai F-score.
%!TEX root = skripsi.tex
%-----------------------------------------------------------------------------%
\chapter{\babLima}
%-----------------------------------------------------------------------------%
Bab ini menjelaskan mengenai hasil yang didapatkan dari eksperimen, serta evaluasi dan analisis terkait hasil tersebut.

%-----------------------------------------------------------------------------%
\section{Korpus Identik}
%-----------------------------------------------------------------------------%
Korpus identik berisi pasangan-pasangan kalimat Indonesia-Inggris sebanyak 88.919 kalimat untuk masing-masing bahasa.
%-----------------------------------------------------------------------------%

%-----------------------------------------------------------------------------%
\section{Pembuatan \textit{Sense Tagged Corpus} Bahasa Inggris}
Tabel \ref{table:sense-tagged-corpus} menunjukan jumlah token (kata) pada korpus bahasa Inggris dan yang diberikan \textit{tag sense} oleh IMS

\begin{table}
	\centering
	\caption{Jumlah \textit{instance} korpus bahasa Inggris}
	\label{table:sense-tagged-corpus}
	\begin{tabular}{|p{0.7cm}|p{4cm}|p{4cm}|}
		\hline
		No & Tipe & Jumlah
		\\ \hline
		1    & 
		Token (kata)   & 
		1.801.484
		\\ \hline
		2    & 
		Kata yang diberikan \textit{tag} oleh IMS     & 
		1.024.797 
		\\ \hline
	\end{tabular}
\end{table}

Berdasarkan proses pembuatan dan hasil dari \textit{sense tagged corpus} tersebut, dapat dilihat bahwa tidak semua kata diberikan \textit{sense} oleh IMS. Kata-kata sapaan seperti "I", "you", dan kata \textit{articles} yaitu "a", "the", "an". Selain itu, kata yang tidak terdapat pada model juga tidak diberi \textit{tag} (dilewati) oleh IMS. Tingkat kebenaran dari \textit{sense tag} yang diberikan bergantung dari model yang digunakan pada penelitian. Terdapat banyak kasus-kasus dimana pemberian \textit{sense} yang dilakukan adalah benar seperti misalnya pada kata "\textit{visitor}" yang diberikan \textit{tag} dengan \textit{sense key} 1:18:00::, yang mana berdasarkan wordnet Princeton "visitor\%1:18:00::" memiliki arti sebagai "\textit{someone who visits}". Contoh lain dari kata yang diberikan \textit{tag} dengan benar adalah "\textit{company}" pada konteks potongan kalimat "\textit{Plantation company PT ...}". Kata "\textit{company}" tersebut diberikan tag "company\%1:14:01::" yang berdasarkan wordnet Princeton memiliki makna "\textit{an institution created to conduct business}". Namun demikian, terjadinya kesalahan pemberian \textit{tag} pada kata terjadi pada kasus-kasus seperti: 

\begin{enumerate}
	\item Sebuah entitas diberikan \textit{tag} dimana entitas tersebut dianggap kata biasa. Contohnya adalah kata "Scotland Yard" dimana "Yard" pada kata tersebut diberikan \textit{tag} yang diartikan sebagai "\textit{a unit of length equal to 3 feet}". Hal ini menunjukan bahwa \textit{tool} belum dapat membedakan antara entitas yang memang tidak perlu diberikan \textit{tag} dan kata biasa (walaupun kata tersebut sudah memiliki huruf kapital).
	\item Kesalahan \textit{tag} dikarenakan \textit{training data} yang digunakan oleh model. Pada potongan kalimat "\textit{... FASB rule will cover such financial instruments as interest rate swaps financial ...}, kata "\textit{interest}" diberikan tag dengan makan "a sense of concern with and curiosity about someone or something". Berdasarkan konteks kalimat tersebut, dapat diketahui bahwa makna yang seharusnya didapat untuk kata "\textit{interest}" diatas ialah "bunga bank". Hal ini sepertinya terjadi karena data yang digunakan untuk \textit{training} model IMS memiliki ketidakseimbangan data untuk model kata "\textit{interest}" sehingga \textit{tag} yang diberikan lebih cenderung kepada "ketertarikan".
	\item Pemberian \textit{tag} pada \textit{multi word} token seperti "\textit{make up}" masih diberikan pada setiap kata. Berdasarkan percobaan untuk \textit{tagging} pada kata tersebut, kata "\textit{make}" dan "\textit{up}" masing-masing diberikan tag yang berbeda. Hal ini terjadi karena IMS mengolah kata demi kata dengan proses tokenisasi \textit{by default} menggunakan spasi. Setelah dilakukan pemeriksaan pada kata-kata yang terdapat pada model, kata \textit{make up} ternyata disimpan sebagai "make\_up". Berdasarkan pemeriksaan tersebut, diperlukan adanya \textit{pre-processing} terlebih dahulu untuk mengganti \textit{separator} kata multiword yang umumnya menggunakan spasi dengan "\_" agar IMS dapat memberikan \textit{tag multi word} tersebut dengan benar. Selain \textit{pre-processing},  IMS juga dapat melakukan \textit{tagging} dengan input dalam format XML. Bentuk kata-kata dan kalimat dalam format XML tersebut biasanya memiliki multi word yang sudah dijadikan satu token sehingga mempermudah penyelesaian masalah tersebut.
\end{enumerate}
%-----------------------------------------------------------------------------%

%-----------------------------------------------------------------------------%
\section{Evaluasi \textit{Word Alignment}}

Hasil dari proses \textit{word alignment} yang dilakukan Giza dibandingkan dengan hasil \textit{alignment} yang dibuat oleh dua orang anotator. Jumlah yang akan dibandingkan adalah 200 buah pasangan data yang didapat dengan \textit{random sampling}. Indikator performa dari perbandingan tersebut adalah nilai dari \textit{precision} dan \textit{recall}.

%-----------------------------------------------------------------------------%
\section{\textit{Sense Transfering}}

Proses \textit{transfer} makna kata dari bahasa Inggris ke bahasa Indonesia yang dilakukan sangat bergantung dari hasil \textit{alignment} kata pada proses sebelumnya. Untuk sebagian besar kata yang memiliki pasangan kata yang benar, proses \textit{transfer} dapat menghasilkan makna yang benar juga. Hal tersebut didukung jika \textit{sense tagged word} pada korpus bahasa Inggris juga benar). Terdapat beberapa kata yang dipilih sebagai \textit{sampling} untuk mengevaluasi hasil \textit{sense transfering}. Kelompok ini dibagi menjadi:

\begin{enumerate}
	\item Jumlah Kelas
	\begin{enumerate}
		\item 3-5 kelas kata
		\item lebih dari 5 kelas kata
	\end{enumerate}
	\item sebaran jumlah \textit{instance} dalam kelasnya
	\begin{enumerate}
		\item \textit{balance}
		\item \textit{imbalance}
	\end{enumerate}
	\item Bentuk morfologi dari kata tersebut
	\begin{enumerate}
		\item Lemma (kata dasar)
		\item Berimbuhan baik itu infleksional ataupun \textit{derivative}
	\end{enumerate}
\end{enumerate}

Pada jumlah kelas sebanyak 3-5 kelas kata (\textit{sense key}), \textit{target word} yang diambil adalah "memecahkan". Kata tersebut memiliki 4 buah kelas total dengan \textit{sense key} yang didapat yaitu 'solve\%2:31:00::','resolve\%2:31:01::', 'break\%2:30:03::', dan 'split\%2:38:00::'. Kata "menolak" mewakili kelas kata sebanyak 10 buah yang diantaranya mengandung kelas 'refuse\%2:32:00::', 'reject\%2:40:00::', 'decline\%2:32:00::', dan beberapa kelas lainnya. Tabel \ref{table:number-classes-sense-transfering-evaluation} menunjukan contoh beberapa kata tersebut dalam beberapa konteks kalimat yang bersesuaian.

\begin{table}
	\centering
	\caption{Evaluasi \textit{Sense Transfering} Berdasarkan Jumlah Kelas}
	\label{table:number-classes-sense-transfering-evaluation}
	\begin{tabular}{|p{4cm}|p{4cm}|p{4cm}|}
		\hline
		\textit{Sense Key} & Makna & Kalimat
		\\ \hline
		solve\%2:31:00::  & 
		\textit{find the solution to (a problem or question) or understand the meaning of}   & 
		salah satu cara untuk \textbf{memecahkan} persoalan yang pelik...
		\\ \hline
		resolve\%2:31:01:: & 
		\textit{bring to an end / settle conclusively}   & 
		evolusionis masih belum bisa \textbf{memecahkan} permasalahan darwin...
		\\ \hline
		break\%2:30:03:: & 
		\textit{terminate}   & 
		...base mereka \textbf{memecahkan} rekor untuk...
		\\ \hline
		split\%2:38:00:: &
		\textit{go one's own way; move apart;} &
		senat mereka \textbf{memecahkan} perbedaan antara skenario 1 dan 3 dengan
		\\ \hline
		decline\%2:32:00:: &
		\textit{show unwillingness towards} &
		wells rich \textbf{menolak} untuk berkomentar...
		\\ \hline
		refuse\%2:32:00:: &
		\textit{show unwillingness towards} &
		kelompok pemberontak yang \textbf{menolak} menandatangani perjanjian ...
		\\ \hline
		reject\%2:40:00:: &
		\textit{refuse to accept} &
		..dua pekan lalu \textbf{menolak} tawaran pemerintah...
		\\ \hline
	\end{tabular}
\end{table}

Makna kata \textit{split} yang diberikan hanya berjumlah satu buah dari keseluruhan korpus, hal ini disebabkan karena kata bahasa Inggris yang digunakan pada kalimat bahasa Inggrisnya menggunakan kata \textit{split}. Berdasarkan \textit{sampling} yang dilakukan, jumlah kelas kata yang ada bergantung pada sebanyak apa sebuah kata di bahasa Indonesia dipasangkan dengan kata bahasa Inggris yang berbeda dan memiliki makna pada \textit{sense tagged english corpus}. Jumlah kelas ini dapat bergantung pada seberapa akurasi \textit{alignment} kata yang dilakukan pada proses sebelumnya.

Pada sebaran jumlah \textit{instance} di dalam kelas-kelasnya, kata "kehadiran" mewakili data yang jumlahnya relatif tidak seimbang. Dimana jumlah kelas \textit{attendance} (19 buah) dan \textit{presence} (64 buah). Kedua \textit{sense} tersebut memiliki makna yang kurang lebih menyatakan sebuah \textit{state} dimana seseorang datang/hadir. Perbandingan jumlah \textit{instance} yang lebih \textit{balance} dari kata "kehadiran" salah satunya adalah kata "rumahnya". \textit{sense key}. Tabel \ref{table:class-instance-sense-transfering-evaluation} menunjukan makna kata yang dipindahkan berdasarkan \textit{sampling} berdasarkan sebaran \textit{instance} dalam kelas.

\begin{table}
	\centering
	\caption{Evaluasi \textit{Sense Transfering} Berdasarkan Sebaran Kelas}
	\label{table:class-instance-sense-transfering-evaluation}
	\begin{tabular}{|p{4cm}|p{2.85cm}|p{2.85cm}|p{1.2cm}|}
		\hline
		Kata (Sense key) & Makna & Kalimat & Jumlah
		\\ \hline
		Kehadiran (attendance\%1:04:00::)  & 
		\textit{the act of being present (at a meeting or event etc.)}   & 
		...tingkat \textbf{kehadiran} guru di sekolah... &
		19
		\\ \hline
		Kehadiran (presence\%1:09:00::) & 
		\textit{the impression that something is present}   & 
		...berkurangnya \textbf{kehadiran} pria dewasa...
		&
		64
		\\ \hline
		rumahnya (house\%1:14:02::) & 
		\textit{an official assembly having legislative powers} & 
		...kebakaran yang melanda \textbf{rumahnya}...
		& 32
		\\ \hline
		rumahnya (home\%1:06:00::) &
		\textit{Housing that someone is living in} &
		...ia pulang ke \textbf{rumahnya} pada sabtu...
		& 19
		\\ \hline
	\end{tabular}
\end{table}

Kedua makna pada kata "kehadiran" memiliki makna yang relatif dekat dan masuk dalam konteks kalimat kata tersebut muncul. Namun demikian, sense key "house\%1:14:02::" tersebut memiliki makna yang salah, dimana sense key yang lebih tepat semestinya adalah house\%1:06:01:: dengan makna "\textit{a building in which something is sheltered or located}". Kesalahan makna kata yang dipindahkan tersebut disebabkan karena kata \textit{house} pada korpus inggris diberikan tag 'house\%1:06:01::', hal ini sepertinya terjadi karena data yang digunakan untuk training model tersebut lebih banyak mengandung kata 'house' dengan makna tersebut.

Kelompok \textit{sampling} lain adalah makna yang akan dilihat pada kata dengan bentuk morfologi yang berbeda. Kata "makan", "makanan", dan "memakan" merupakan kata yang mewakili kasus bentuk morfologi dalam bentuk lemma maupun berimbuhan. Pada kata "makan", \textit{sense key} yang diterima dari hasil \textit{transfer} adalah eat\%2:34:00:: yang memiliki makna "\textit{take in solid food}". Kata "makanan" pada kalimat-kalimat yang ada diberikan \textit{sense key} food\%1:03:00:: yang diartikan sebagai "\textit{any substance that can be metabolized by an animal to give energy and build tissue}". Kata "memakan" sendiri memiliki beberapa \textit{sense key} seperti consume\%2:34:02::, eat\%2:34:00::, dan feed\%1:13:00::. Dari \textit{sense key} yang didapat tersebut, consume\%2:34:02::("spend extravagantly") bukan merupakan \textit{sense} yang tepat (seharusnya memiliki makna mengonsumsi makanan), dan feed\%1:13:00:: ("\textit{food for domestic livestock}") yang semestinya adalah "memberikan makanan". Makna kata yang tidak tepat pada hasil-hasil tersebut merupakan kesalahan dari baik itu variasi \textit{alignment} suatu kata yang dipasangkan dengan kata lainnya, dan juga model IMS yang kurang pas dengan domain data pada penelitian ini.

%-----------------------------------------------------------------------------%
\section{Sistem WSD}

Untuk melihat seberapa baik performa sistem WSD dengan menggunakan \textit{sense tagged corpus} hasil dari penelitian, terdapat kata-kata yang dipilih secara manual sebagai sampling dari \textit{target word} yang akan dievaluasi berdasarkan nilai F-score dari hasil rata-rata \textit{cross validation}. Kata yang dipilih merupakan \textit{instance} yang memiliki pasangan kata lebih dari satu dalam bahasa Inggris dan mempunyai makna yang berbeda dari hasil \textit{sense transfering}. \textit{Target word} yang dipilih tersebut memiliki kriteria bahwa pasangan bahasa inggris kata tersebut lebih dari satu ataupun juga pasangan bahasa inggrisnya memiliki makna yang tidak dekat (ambigu). Fitur yang dilakukan percobaan pada penelitian ini adalah F1(\textit{bag of words}), F2 (\textit{word embedding}), F3 (\textit{pos-tag}), F4(\textit{pos tagging} dan \textit{bag of words}). Terdapat dua tabel dengan perbedaan berupa kamus yang digunakan dari proses \textit{enhancement} sebelumnya. Hasil evaluasi dapat dilihat pada tabel akurasi  \ref{table:wsd-evaluation-crawling} dan \ref{table:wsd-evaluation-bidirectional} berikut.

\begin{table}
	\centering
	\caption{Evaluasi sistem WSD kamus \textit{crawling}}
	\label{table:wsd-evaluation-crawling}
	\begin{tabular}{|p{3cm}|p{1.5cm}|p{1.5cm}|p{1.5cm}|p{1.5cm}|p{1.5cm}|}
		\hline
		Kata & Baseline & f1 & f2 & f3 & f4 \\ \hline
		meninggalkan & 0.89 & 0.89 & 0.88 & 0.89 & 0.88 \\ \hline
		memecahkan & 0.5 & 0.5 & 0.54 & 0.5 & 0.46 \\ \hline
		menolak & 0.6 & 0.65 & 0.6 & 0.75 & 0.76 \\ \hline
		obat & 0.49 & 0.7 & 0.56 & 0.61 & 0.78 \\ \hline
		lingkungan & 0.54 & 0.54 & 0.43 & 0.67 & 0.66 \\ \hline
		halaman & 0.93 & 0.93 & 0.93 & 0.91 & 0.93 \\ \hline
		kehadiran & 0.67 & 0.91 & 0.77 & 0.73 & 0.91 \\ \hline
		hati & 0.72 & 0.84 & 0.77 & 0.73 & 0.84 \\ \hline
		coklat & 0.33 & 0.61 & 0.5 & 0.22 & 0.39 \\ \hline
		acara & 0.58 & 0.6 & 0.53 & 0.45 & 0.55 \\ \hline
		berat & 0.53 & 0.61 & 0.52 & 0.68 & 0.7 \\ \hline
		jalan & 0.66 & 0.75 & 0.73 & 0.68 & 0.74 \\ \hline
	\end{tabular} 
\end{table}


\begin{table}
	\centering
	\caption{Evaluasi sistem WSD kamus \textit{bi-directional}}
	\label{table:wsd-evaluation-bidirectional}
	\begin{tabular}{|p{3cm}|p{1.5cm}|p{1.5cm}|p{1.5cm}|p{1.5cm}|p{1.5cm}|}
		\hline
		Kata & Baseline & f1 & f2 & f3 & f4 \\ \hline
		meninggalkan & 0.9 & 0.9 & 0.9 & 0.9 & 0.9 \\ \hline
		memecahkan & 0.46 & 0.46 & 0.46 & 0.38 & 0.42 \\ \hline
		menolak & 0.63 & 0.65 & 0.57 & 0.75 & 0.76 \\ \hline
		obat & 0.5 & 0.58 & 0.54 & 0.45 & 0.55 \\ \hline
		lingkungan & 0.55 & 0.55 & 0.45 & 0.67 & 0.68 \\ \hline
		halaman & 0.85 & 0.9 & 0.85 & 0.77 & 0.88 \\ \hline
		kehadiran & 0.68 & 0.85 & 0.75 & 0.67 & 0.81 \\ \hline
		hati & 0.75 & 0.86 & 0.85 & 0.75 & 0.86 \\ \hline
		coklat & 0.17 & 0.72 & 0.5 & 0.33 & 0.67 \\ \hline
		acara & 0.53 & 0.55 & 0.53 & 0.53 & 0.55 \\ \hline
		berat & 0.49 & 0.63 & 0.52 & 0.68 & 0.68 \\ \hline
		jalan & 0.68 & 0.76 & 0.71 & 0.71 & 0.76 \\ \hline
	\end{tabular} 
\end{table}



\begin{table}
	\centering
	\caption{Evaluasi Word Alignment}
	\label{table:word-alignment-evaluation}
	\begin{tabular}{|p{2cm}|p{2cm}|p{2cm}|p{2cm}|}
		\hline
		Anotator & Precision & Recall & F-Score
		\\ \hline
		1 & 0.775 & 0.747 & 0.761
		\\ \hline
		2 & 0.768 & 0.75 & 0.759
		\\ \hline
	\end{tabular} 
\end{table}

Pada kata "halaman" persebaran \textit{instance} dalam kelasnya adalah page\%1:10:00:: sebanyak 41 buah, yard\%1:23:00:: 3 buah, dan courtyard\%1:06:00:: 3 buah. Jumlah data yang tidak \textit{balance} pada kelas-kelas tersebut bisa jadi merupakan penyebab tingginya akurasi yang ada bahkan pada tingkat \textit{baseline}. Berbeda dengan kata "kehadiran" yang memiliki persebaran makna kata attendance\%1:04:00:: sebanyak 19 buah, presence\%1:09:00:: 64 buah, dan existence\%1:26:00:: 1 buah. Dengan perbedaan antara kelas makna 'attendance' dan 'presence' yang tidak sejauh pada kasus kata 'halaman', akurasi baseline yang dihasilkan lebih rendah dari baseline 'halaman'. Namun demikian, akurasi dari sistem WSD yang dihasilkan untuk kata 'kehadiran' tersebut diatas baseline untuk semua fitur yang dicoba.
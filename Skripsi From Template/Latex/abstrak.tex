%!TEX root = skripsi.tex
%
% Halaman Abstrak
%
% @author  Andreas Febrian
% @version 1.00
%

\chapter*{Abstrak}

\vspace*{0.2cm}

\noindent \begin{tabular}{l l p{10cm}}
	Nama&: & \penulis \\
	Program Studi&: & \program \\
	Judul&: & \judul \\
\end{tabular} \\ 

\vspace*{0.5cm}

\noindent
\textit{Textual Entailment} adalah penelitian di bidang NLP yang bertujuan untuk mengidentifikasikan apakah terdapat hubungan \textit{entailment} di antara dua buah teks. Penelitian \textit{Textual Entailment} sudah dikembangkan dalam berbagai bahasa, namun \textit{Textual Entailment} untuk Bahasa Indonesia masih sangat minim. Penelitian ini ditujukan untuk mengembangkan korpus \textit{Textual Entailment} Bahasa Indonesia secara otomatis menggunakan metode Co-training, sebuah metode \textit{semi-supervised learning} yang pernah digunakan pada pengembangan korpus \textit{Textual Entailment} Bahasa Inggris. Sumber data yang digunakan untuk Co-training adalah Wikipedia \textit{revision history}. Pada akhir penelitian, terdapat sejumlah 1857 data korpus yang dihasilkan secara otomatis dengan akurasi data sebesar adalah 76\%. Hasil tersebut menunjukkan bahwa kombinasi metode Co-training dan data Wikipedia \textit{revision history} berpotensi menghasilkan korpus \textit{Textual Entailment} yang berukuran besar dan baik.


\vspace*{0.2cm}

\noindent Kata Kunci: \\ 
\noindent \textit{Textual Entailment}, Co-training, Wikipedia \textit{revision history}, korpus, Bahasa Indonesia\\

\newpage
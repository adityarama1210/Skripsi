%!TEX root = skripsi.tex
%
% Halaman Abstrak
%
% @author  Andreas Febrian
% @version 1.00
%

\chapter*{Abstrak}

\vspace*{0.2cm}

\noindent \begin{tabular}{l l p{10cm}}
	Nama&: & \penulis \\
	Program Studi&: & \program \\
	Judul&: & \judul \\
\end{tabular} \\ 

\vspace*{0.5cm}

\noindent
\textit{Cross Lingual Word Sense Disambiguation (CLWSD)} merupakan salah satu pendekatan untuk menyelesaikan permasalahan disambiguasi makna kata di bidang NLP. Pendekatan ini memanfaatkan sebuah konsep di mana suatu kata dapat diterjemahkan menjadi beberapa kata yang berbeda tergantung dengan konteks di mana kata tersebut muncul. Keterbatasan data berupa \textit{sense tagged corpus} menjadi salah satu permasalahan yang menghambat penelitian WSD di bahasa Indonesia ini. Pada penelitian kali ini, pendekatan CLWSD akan digunakan untuk \textit{transfer sense} dari \textit{sense tagged corpus} bahasa Inggris ke bahasa Indonesia dengan memanfaatkan korpus paralel dwibahasa. Hasil dari penelitian ini merupakan \textit{sense tagged corpus} dalam bahasa Indonesia yang kemudian juga akan dicoba dalam sistem WSD yang dibangun sendiri. Pada penelitian ini, \textit{sense transferring} berhasil menghasilkan \textit{sense tagged corpus} dalam bahasa Indonesia Evaluasi pada \textit{sample} pada \textit{sense tagged corpus} yang dihasilkan tersebut memiliki akurasi (precision) kesesuaian makna sebesar 84.8\%. Selain \textit{sense tagged corpus} tersebut, sistem WSD yang dibangun juga memiliki performa di atas baseline pembanding pada \textit{target word} dan fitur-fitur yang dicoba. Pada kombinasi fitur \textit{bag of words} dan \textit{POS Tag}, sistem WSD Bahasa Indonesia tersebut menghasilkan F-Score terbaik sebesar 0.682 (68.2\%).


\vspace*{0.2cm}

\noindent Kata Kunci: \\ 
\noindent Cross Lingual, Word Sense Disambiguation

\newpage
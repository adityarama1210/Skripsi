%!TEX root = skripsi.tex
%
% Halaman Abstrak
%
% @author  Andreas Febrian
% @version 1.00
%

\chapter*{Abstrak}

\vspace*{0.2cm}

\noindent \begin{tabular}{l l p{10cm}}
	Nama&: & \penulis \\
	Program Studi&: & \program \\
	Judul&: & \judul \\
\end{tabular} \\ 

\vspace*{0.5cm}

\noindent
\textit{Cross Language Word Sense Disambiguation (CLWSD)} merupakan salah satu pendekatan untuk menyelesaikan permasalahan disambiguasi makna kata di bidang NLP. Pendekatan ini memanfaatkan sebuah konsep dimana suatu kata dapat diterjemahkan menjadi beberapa kata yang berbeda tergantung dengan konteks dimana kata tersebut muncul. Keterbatasan data berupa \textit{sense tagged corpus} menjadi salah satu permasalahan yang menghambat penelitian WSD di bahasa Indonesia ini. Pada penelitian kali ini, \saya~akan menggunakan pendekatan CLWSD untuk \textit{transfer sense} dari \textit{sense tagged corpus} bahasa Inggris ke bahasa Indonesia dengan memanfaatkan korpus paralel dwibahasa. Hasil dari penelitian ini merupakan \textit{sense tagged corpus} dalam bahasa Indonesia yang kemudian juga akan dicoba dalam sistem WSD yang dibuat sendiri. Berdasarkan hasil percobaan, dapat terlihat bahwa korpus hasil dari \textit{sense transfering} menghasilkan data yang banyak dengan kualitas cukup baik. Selain \textit{sense tagged corpus} tersebut, sistem WSD yang dibangun juga memiliki performa diatas baseline pembanding pada \textit{target word}  tertentu yang diberikan.


\vspace*{0.2cm}

\noindent Kata Kunci: \\ 
\noindent Cross Language, Word Sense Disambiguation

\newpage